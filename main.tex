%! TEX TS-program = LuaLaTeX

% Gemini theme
% See: https://rev.cs.uchicago.edu/k4rtik/gemini-uccs
% A fork of https://github.com/anishathalye/gemini

\documentclass[12pt, usenames, dvipsnames]{beamer}

% ====================
% Packages
% ====================

\usepackage[T1]{fontenc}
\usepackage{lmodern}
\usepackage[size=a0, scale=1]{beamerposter}
\usetheme{gemini}
\usecolortheme{stanford}

\usepackage{graphicx}
\usepackage{booktabs}

\usepackage{hyperref}
\usepackage{amsthm}
\usepackage{thmtools}

% For theorems and such
\usepackage{amsmath}
\usepackage{amssymb}
\usepackage{mathtools}

\usepackage{natbib}

\usepackage{xcolor}
\usepackage{algorithm}
\usepackage{algorithmic}
\usepackage{bbm}

% ====================
% Lengths
% ====================

% If you have N columns, choose \sepwidth and \colwidth such that
% (N+1)*\sepwidth + N*\colwidth = \paperwidth
\newlength{\sepwidth}
\newlength{\colwidth}
\setlength{\sepwidth}{0.025\paperwidth}
\setlength{\colwidth}{0.3\paperwidth}

\newcommand{\separatorcolumn}{\begin{column}{\sepwidth}\end{column}}

\definecolor{DarkFern}{HTML}{407428}
\colorlet{Fern}{DarkFern!85!white}
\colorlet{LightFern}{DarkFern!20!white}
\setbeamercolor{result}{bg=LightFern}

%% load custom macros and pre-defined symbols
\input{macros/math}
%% macros for paragraph mode
\input{macros/paragraph}

%% tikz

\usepackage{tikz}
\usepackage{pgfplots}
\pgfplotsset{compat=1.16}
\input{macros/plots}
\usetikzlibrary{shapes, arrows}
\usetikzlibrary{decorations.pathreplacing, calligraphy}

\newcommand{\red}[1]{\textcolor{Red}{#1}}
\newcommand{\green}[1]{\textcolor{ForestGreen}{#1}}
\newcommand{\blue}[1]{\textcolor{DarkBlue}{#1}}
\newcommand{\purple}[1]{\textcolor{Magenta}{#1}}

% node styles
\tikzstyle{Input}=[minimum size=0.3cm, fill=black, line width = 0.5mm, draw=black, shape=circle, text=black]
\tikzstyle{Hidden}=[minimum size=0.3cm, fill=blue, line width = 0.5mm, draw=blue, shape=circle, text=black]
\tikzstyle{Splits}=[inner sep=0.03cm, minimum size=0.3cm, line width = 0.3mm, draw=blue, shape=circle, text=black]
\tikzstyle{Output}=[minimum size=0.3cm, fill=white, line width = 0.5mm, draw=black, shape=circle, text=black]

% Edge styles
\tikzstyle{arrow}=[line width = 0.5mm]


\title{Fast Convex Optimization for Two-Layer ReLU Networks: \\ {\huge Equivalent Model Classes and Cone Decompositions}}


\author{Aaron Mishkin \inst{1} \and Arda Sahiner \inst{2} \and Mert Pilanci \inst{2}}

\institute[shortinst]{\inst{1} Department of Computer Science, Stanford University \samelineand \inst{2} Department of Electrical Engineering, Stanford University}

% ====================
% Footer (optional)
% ====================

\footercontent{
	\href{http://github.com/pilancilab/scnn}{http://github.com/pilancilab/scnn} \hfill
	ICML 2022 \hfill
	\href{mailto:amishkin@cs.stanford.edu}{amishkin@cs.stanford.edu}}
% (can be left out to remove footer)

% ====================
% Logo (optional)
% ====================

% use this to include logos on the left and/or right side of the header:
% \logoright{\includegraphics[height=7cm]{logos/cs-logo-maroon.png}}
% \logoleft{\includegraphics[height=7cm]{logos/cs-logo-maroon.png}}

% ====================
% Body
% ====================

\begin{document}

% This adds the Logos on the top left and top right
\addtobeamertemplate{headline}{}
{
	\begin{tikzpicture}[remember picture,overlay]
		%   \node [anchor=north west, inner sep=3cm] at ([xshift=0.0cm,yshift=1.0cm]current page.north west)
		%   {\includegraphics[height=5.0cm]{stanford_logos/Stanford-CS.png}}; % uc-logo-white.eps
		\node [anchor=north east, inner sep=3cm] at ([xshift=0.0cm,yshift=2.5cm]current page.north east)
		{\includegraphics[height=10.0cm]{stanford_logos/Block_S_2_color.png}};
	\end{tikzpicture}
}

\begin{frame}[t]
	\begin{columns}[t]
		\separatorcolumn

		\begin{column}{\colwidth}
			\vspace{-1.5em}
			\begin{block}{Introduction}
				\large
				{\large \red{Problem}: optimizing neural networks with stochastic gradient methods is hard.}

				\begin{itemize}
					\item \textbf{Tuning}: good performance requires tuning the step-size, momentum, \ldots

					\item \textbf{Model Churn}: changing extrinsic parameters like random seed affects model performance \citep{henderson2018deep}.

					\item \textbf{Certificates}: convergence to stationary point, but only with decreasing step-sizes.
				\end{itemize}

				{\large \green{Approach}: train two-layer models by reformulating them as convex programs.}
				\vspace{-0.25em}

				\begin{figure}[]
					\centering
					\includegraphics[width=.9\textwidth]{assets/synthetic_classification.png}
					\caption{Effect of random seed on convergence of SGD for a realizable problem.}
					\vspace{-1em}
					\label{fig:synthetic_classification}
				\end{figure}

			\end{block}

			\begin{block}{Convex Reformulations: ReLU Networks}

				\begin{columns}[t]
					\begin{column}{0.62\textwidth}

						\large
						{\Large \red{Non-Convex Problem}:}
						\vspace{1em}
						\[
							\Large
							\min_{w^1, w^2} \underbrace{\norm{\sum_{j=1}^m (X w^1_j)_+ w_j^2 - y}_2^2}_{\text{Squared Error}}
							+ \underbrace{\lambda \sum_{j=1}^m \norm{w^1_j}_2^2 + \norm{w^2_j}_2^2}_{\text{Weight Decay}},
						\]
						where \( \rbr{x}_+ = \max\cbr{x, 0} \) is the ReLU activation.


						\vspace{2.5em}
						\hrule
						\vspace{2.5em}

						{\Large \green{Convex Reformulation}}: {\normalsize \citep{pilanci2020convexnn}}
						\vspace{1em}
						\[ \Large
							\begin{aligned}
								\min_{v, h} & \norm{\sum_{j=1}^p D_j X (v_j - h_j) - y}_2^2 +
								\lambda \sum_{j=1}^p \norm{v_j}_2 + \norm{h_j}_2              \\
								            & \hspace{0.2em} \purple{\text{s.t. }
									v_j, h_j \in \calK_j := \cbr{w : (2D_j - I) X w \geq 0}},
							\end{aligned}
						\]

						\vspace{1em}
						where \( D_j = \text{diag}[\mathbbm{1}(X g_j \geq 0)] \).

					\end{column}

					\begin{column}{0.55\textwidth}
						\vspace{-2em}

						\begin{figure}[]
							\centering
							\input{assets/neural_net}
						\end{figure}

						\vspace{-3em}
						\begin{figure}[]
							\centering
							\input{assets/convex_reformulation}
						\end{figure}

					\end{column}

				\end{columns}

			\end{block}


		\end{column}

		\separatorcolumn

		\begin{column}{\colwidth}

			\vspace{-1.5em}
			\begin{block}{Problem Scale}
				\large

				The \textbf{convex program} enumerates all activation patterns,
				\vspace{0.5em}
				\[
					p = \abs{\cbr{D_j = \text{diag}[\mathbbm{1}(X g_j \geq 0)] : g_j \in \R^d }}.
				\]
				\vspace{-1em}
				\begin{itemize}
					\item \red{Exponential in general}: \( p \in O(r \cdot (\frac{n}{r})^r) \),
					      where \( r = \text{rank}(X) \) \citep{winder1966partitions}.
					      \begin{itemize}
						      \item \large But sub-sampling works well in practice.
					      \end{itemize}
					      \vspace{0.1em}

					\item \green{Highly structured} --- it's a huge-scale constrained generalized linear model!

					\item The convex reformulation \purple{exchanges one kind of hardness for another}.
				\end{itemize}
			\end{block}

			\vspace{-1em}
			\begin{block}{Gated ReLU Activations}
				\large

				\red{Issue}: convex problem is too large for IPMs, but projecting onto \( \calK_j \) is an LP.

				\green{Solution}: consider unconstrained relaxation:
				\[
					\begin{aligned}
						\textbf{C-GReLU}: \min_{u} & \norm{\sum_{j=1}^p D_j X u_j - y}_2^2 +
						\lambda \sum_{j=1}^p \norm{u_j}_2                                    \\
					\end{aligned}
				\]

				\begin{beamercolorbox}[wd=\textwidth,sep=1em]{result}
					\textbf{Theorem 2.2} (informal): C-GReLU is equivalent to solving
					\[
						\textbf{NC-GReLU}: \min_{w^1, w^2} \half \norm{\sum_{j=1}^p \phi_{g_j}(X, w^1_{j})w^2 - y}_2^2 + \frac{\lambda}{2} \sum_{j=1}^p \norm{w^1_{j}}_2^2 + \norm{w_j^2}_2^2,
					\]
					with the ``Gated ReLU'' \citep{fiat2019decoupling} activation function
					\[ \phi_{g}(X, u) = \text{diag}(\mathbbm{1}(Xg \geq 0)) X u, \]
					and gate vectors \( g_j \) such that \( D_j = \text{diag}[\mathbbm{1}(X g_j \geq 0) \).
				\end{beamercolorbox}

				\textbf{Interpretation}: if \( u_j \not \in \calK_j \), then activation
				must be decoupled from weights.

			\end{block}
			\vspace{-1em}
			\begin{block}{Cone Decompositions}
				\large
				{ \Large
					\red{Q}: when are Gated ReLU and ReLU networks equivalent?

					\green{A}: if we can decompose \( u_j = v_j - h_j \) for some \( v_j, h_j \in \calK_j \).
				}

				\begin{figure}[t]
					%! TEX root = ../../main.tex

%% Illustration of cone decomposition. 

\begin{tikzpicture}[scale=1,
		declare function={
				cone_u(\x)= -\x/3;
				cone_l(\x)= -4*\x/5;
			}
	]
	\begin{axis}[width=0.9\linewidth, height=14cm,
			axis lines=center, yticklabels={,,}, xticklabels={,,},
			ymin=-4, ymax=4, ytick={-5,...,5}, ylabel=$$, x axis line style={-},
				xmin=-6, xmax=6, xtick={-5,...,5}, xlabel=$$, y axis line style={-},
		]
		\addplot[name path=cone_u, domain=-6:6, samples=100, line width=6pt]{cone_u(x)};
		\addplot[name path=cone_l, domain=-6:6, samples=200, line width=6pt]{cone_l(x)};

		% add color fill to both cones.
		\addplot fill between[
				of = cone_u and cone_l,
				split, % calculate segments
				every even segment/.style = {fill=blue, fill opacity=0.3},
				every odd segment/.style  = {fill=teal, fill opacity=0.3}
			];

		%% point labels
		% origin point
		\node[circle, fill, inner sep=5pt] at (axis cs:0,0) {};

		\node[label={0:\Large $u_j$}, circle, fill, inner sep=8pt] (u) at (axis cs:2,1) {};
		\node[label={90:\Large $v_j$}, circle, fill, inner sep=8pt] (v) at (axis cs:25/7+2, -25/21 - 2/3) {};
		\node[label={90:\Large $w_j$}, circle, fill, inner sep=8pt] (w) at (axis cs:-25/7, 20/7) {};

		% labels
		\node[label={0:\huge$\calK_j$}] at (axis cs:4,-2.5) {};
        \node[label={180:\huge$-\calK_j$}] at (axis cs:-3.75,2.5) {};

		% lines
		\draw [->, dotted, draw=red, line width = 2.5mm] (u) edge (w);
		\draw [->, dotted, draw=red, line width = 2.5mm] (u) edge (v);
	\end{axis}

\end{tikzpicture}%

					\caption{Illustration of cone decomposition procedure.}
					\label{fig:cone-decomp}
				\end{figure}

				\textbf{Existence} of ``cone decompositions'':
				\vspace{-0.5em}
				\begin{itemize}
					\item We can guarantee \( \calK_j - \calK_j = \R^d \) if \( X \) is \green{full row-rank}.
					\item More generally, we show ``flat'' \( \calK_j \) can be \green{safely merged} into other neurons.
				\end{itemize}
			\end{block}
		\end{column}

		\separatorcolumn

		\begin{column}{\colwidth}
			\vspace{-1.5em}
			\begin{block}{Main Approximation Result}

				\large
				\vspace{1em}

				\begin{beamercolorbox}[wd=\textwidth,sep=1em]{result}
					\textbf{Theorem 3.7} (informal):
					Let \( \lambda \geq 0 \) and let \( p^* \) be the optimal value of the ReLU problem.
					There exists a C-GReLU problem with minimizer \( u^* \) and optimal value \( d^* \) satisfying,
					%\vspace{-1em}
					{ \Large
							\[
								d^* \leq p^* \leq d^* + \textcolor{Red}{2 \lambda \kappa(\tilde X_{\calJ}) \sum_{D_i \in \tilde \calD} \norm{u_i^*}_2}.
							\]
						}
				\end{beamercolorbox}
				\textbf{Consequence}: ReLU and Gated ReLU model classes are equivalent!

			\end{block}
			\vspace{-1em}
			\begin{block}{Algorithms}

				\large
				We develop two algorithms for solving the convex reformulations:
				\vspace{-0.5em}
				\begin{itemize}
					\item \textbf{R-FISTA}: a restarted FISTA variant for Gated ReLU.
					\item \textbf{AL}: an augmented Lagrangian method for the (constrained) ReLU Problem.
				\end{itemize}

				And we can use all the \green{convex tricks}!
				\vspace{-0.5em}
				\begin{itemize}
					\item \textbf{Fast}: \( O(1/t^2) \) convergence rate.

					\item \textbf{Tuning-free}: line-search, restarts, data normalization, \ldots

					\item \textbf{Certificates}: termination based on minimum-norm subgradient.
				\end{itemize}
			\end{block}
			\vspace{-1em}
			\begin{block}{Experiments}
				\large
				\begin{figure}[]
					\centering
					\includegraphics[width=\textwidth]{assets/pp_main.pdf}
					\caption{Performance profile comparing convex solvers to Adam and SGD.}
					\label{fig:performance-profile}
				\end{figure}
				\vspace{-0.5em}
				\begin{itemize}
					\item Performance on 438 training problems generated from the UCI repository.
					\item R-FISTA/AL solve \textbf{more problems, faster}, than SGD and Adam.
				\end{itemize}
			\end{block}
			\vspace{-1em}
			\begin{block}{References}

				\footnotesize{
					\bibliography{
						bib/monographs.bib,
						bib/gradient_methods.bib,
						bib/lower_bounds.bib,
						bib/assumptions.bib,
						bib/interpolation.bib,
						bib/step_sizes.bib,
						bib/datasets.bib,
						bib/general_refs.bib}
				}
				\bibliographystyle{icml2022}

			\end{block}

		\end{column}

		\separatorcolumn
	\end{columns}
\end{frame}

\end{document}

%! TEX root = ../../main.tex

%% Illustration of cone decomposition. 

\begin{tikzpicture}[scale=1,
		declare function={
				cone_u(\x)= -\x/3;
				cone_l(\x)= -4*\x/5;
			}
	]
	\begin{axis}[width=0.9\linewidth, height=14cm,
			axis lines=center, yticklabels={,,}, xticklabels={,,},
			ymin=-4, ymax=4, ytick={-5,...,5}, ylabel=$$, x axis line style={-},
				xmin=-6, xmax=6, xtick={-5,...,5}, xlabel=$$, y axis line style={-},
		]
		\addplot[name path=cone_u, domain=-6:6, samples=100, line width=6pt]{cone_u(x)};
		\addplot[name path=cone_l, domain=-6:6, samples=200, line width=6pt]{cone_l(x)};

		% add color fill to both cones.
		\addplot fill between[
				of = cone_u and cone_l,
				split, % calculate segments
				every even segment/.style = {fill=blue, fill opacity=0.3},
				every odd segment/.style  = {fill=teal, fill opacity=0.3}
			];

		%% point labels
		% origin point
		\node[circle, fill, inner sep=5pt] at (axis cs:0,0) {};

		\node[label={0:\Large $u_j$}, circle, fill, inner sep=8pt] (u) at (axis cs:2,1) {};
		\node[label={90:\Large $v_j$}, circle, fill, inner sep=8pt] (v) at (axis cs:25/7+2, -25/21 - 2/3) {};
		\node[label={90:\Large $w_j$}, circle, fill, inner sep=8pt] (w) at (axis cs:-25/7, 20/7) {};

		% labels
		\node[label={0:\huge$\calK_j$}] at (axis cs:4,-2.5) {};
        \node[label={180:\huge$-\calK_j$}] at (axis cs:-3.75,2.5) {};

		% lines
		\draw [->, dotted, draw=red, line width = 2.5mm] (u) edge (w);
		\draw [->, dotted, draw=red, line width = 2.5mm] (u) edge (v);
	\end{axis}

\end{tikzpicture}%
